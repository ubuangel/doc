\documentclass[12pt]{article}
\usepackage{caption}

\usepackage[utf8]{inputenc}
\usepackage[T1]{fontenc}


\usepackage[spanish]{babel}
\usepackage{titlesec}
\usepackage{titling}
\usepackage{xcolor}
\usepackage{hyperref}%compilar 2 veces
\usepackage{geometry}
\usepackage{listings}
\usepackage{pdfpages}
\usepackage{subfig}%figuras una alado de otraa
\usepackage{float}
\definecolor{commentsColor}{rgb}{0.13, 0.55, 0.13}
\definecolor{keywordsColor}{rgb}{0.000000, 0.000000, 0.635294}
\definecolor{stringColor}{rgb}{0.558215, 0.000000, 0.135316}
\definecolor{numerolineas}{rgb}{0.41,0.41,0.41}

\usepackage{verbatim}%coemntario begin{} end

\usepackage{fancyhdr}%activar para usar encabezados esttilos funcy


%\usepackage[backend=bibtex]{biblatex}
%\addbibresource{referencias/referencias.bib}

%\usepackage[top=1.5cm,bottom=1.0cm,left=1.25cm,right=1.25cm]{geometry}%para todos las paginas


\lstset{ %
  backgroundcolor=\color{white},   % choose the background color; you must add \usepackage{color} or \usepackage{xcolor}
  %large
  basicstyle=\footnotesize,        % the size of the fonts that are used for the code
  breakatwhitespace=false,         % sets if automatic breaks should only happen at whitespace
  breaklines=true,                 % sets automatic line breaking
  captionpos=b,                    % sets the caption-position to bottom
  commentstyle=\color{commentsColor}\textit,    % comment style
  deletekeywords={...},            % if you want to delete keywords from the given language
  escapeinside={\%*}{*)},          % if you want to add LaTeX within your code
  extendedchars=true,              % lets you use non-ASCII characters; for 8-bits encodings only, does not work with UTF-8
  frame=tb,	                   	   % adds a frame around the code
  keepspaces=true,                 % keeps spaces in text, useful for keeping indentation of code (possibly needs columns=flexible)
  keywordstyle=\color{keywordsColor}\bfseries,       % keyword style
  language=Python,                 % the language of the code (can be overrided per snippet)
  otherkeywords={*,...},           % if you want to add more keywords to the set
  numbers=left,                    % where to put the line-numbers; possible values are (none, left, right)
  numbersep=5pt,                   % how far the line-numbers are from the code
  numberstyle=\tiny\color{black}, % the style that is used for the line-numbers
  rulecolor=\color{black},         % if not set, the frame-color may be changed on line-breaks within not-black text (e.g. comments (green here))
  showspaces=false,                % show spaces everywhere adding particular underscores; it overrides 'showstringspaces'
  showstringspaces=false,          % underline spaces within strings only
  showtabs=false,                  % show tabs within strings adding particular underscores
  stepnumber=1,                    % the step between two line-numbers. If it's 1, each line will be numbered
  stringstyle=\color{stringColor}, % string literal style
  tabsize=2,	                   % sets default tabsize to 2 spaces
  title=\lstname,                  % show the filename of files included with \lstinputlisting; also try caption instead of title
  columns=fixed                    % Using fixed column width (for e.g. nice alignment)
}




%activar 
\fancyhf{}
\lhead[\leftmark]{\textit{Angel Andres Bejar Merma}}
\rhead[Nombre Autor]{\rightmark}
\lfoot[\thepage]{}
\rfoot[]{\thepage}
\renewcommand{\headrulewidth}{0.5pt}
\renewcommand{\footrulewidth}{0pt}
%activar 
%\pagestyle{fancy}


\hypersetup{
    colorlinks=true,
    linkcolor=blue,
    filecolor=magenta,      
    urlcolor=cyan,
}



\lstset{literate=
  {á}{{\'a}}1
  {é}{{\'e}}1
  {í}{{\'i}}1
  {ó}{{\'o}}1
  {ú}{{\'u}}1
  {Á}{{\'A}}1
  {É}{{\'E}}1
  {Í}{{\'I}}1
  {Ó}{{\'O}}1
  {Ú}{{\'U}}1
  {ñ}{{\~n}}1
  {ü}{{\"u}}1
  {Ü}{{\"U}}1
}




\begin{document}
%\newgeometry{bottom=2.5cm,top=2.6cm,left=2.5cm,right=2.5cm}
%\restoregeometry
%\includepdf{9}

%
%para el report 
\begin{titlepage}

	\centering
	\includegraphics[width=0.15\textwidth]{imagenes/unsa.png}\par\vspace{1cm}
	{\scshape\LARGE Universidad Nacional de San Agustin \par}
	\vspace{1cm}
	{\scshape\Large Proyecto \par}
	\vspace{1.5cm}
	{\huge\bfseries Titulo\par}
	\vspace{2cm}
	{\Large\itshape Bejar Merma Angel Andres\par}
	\vfill
	Docente \par
	Dr.  \textsc{Nombre de lprofesor}

	\vfill

% Bottom of the page
	{\large \today\par}
\end{titlepage}

%para el report



%\title{Practica numero 6}
%\author{\textit{Angel Andres Bejar Merma}}
%\date{30 de julio del 2020}
%\maketitle

%\begin{abstract}



%\end{abstract}

%\vspace{5mm} %espacio vertical usar en imagen



\section{Parte 1}

Desarrollar un compilador descendente recursivo para evaluar expresiones booleanas,
dada la siguiente gramática.\ref{fig:1}


\lstinputlisting[]{tt.f95}

\begin{figure}[H]%estricto
\centering
\includegraphics[width=10cm]{imagenes/.png}
\caption{Gramatica}
\label{fig:1}
\end{figure}




\begin{figure}[H]%primero aqui sino arriba sino abajo
\centering
\includegraphics[width=15cm]{imagenes/.png}
\caption{}
\end{figure}
\newpage


\section{Desarrollo}
\newpage



\section{Codigo}
\begin{lstlisting}
#include <stdio.h>
#include <stdbool.h>
#include <math.h>
#include<ctype.h>
#include<string.h>

	if (strcmp(lexema, "false")==0)
			return token=false;
	if (strcmp(lexema, "true")==0)
			return token=true;

		}


		}


\end{lstlisting}

%\end{samepage}

\begin{figure}[H]
\centering
\includegraphics[width=15cm]{imagenes/.png}

\caption{}
\end{figure}
Se modifico el codigo 




\begin{figure}[H]
\centering
\includegraphics[width=15cm]{imagenes/.png}
\caption{}
\end{figure}


\newpage
\section{Anexo}



\begin{figure}[H]
\centering
%\includegraphics[width=15cm]{}

\caption{}
\end{figure}

%\includegraphics[width=15cm]{índice2.png} 




\newpage
\section{Conclusion}
%Este algoritmo conocido como el de construcción de
%subconjuntos convierte un AFN en un AFD haciendo uso de




\vspace{20 mm}
	%\section{Conclusion}

%	Esto esposible debido a que C++ es  







%\printbibliography

\end{document}

